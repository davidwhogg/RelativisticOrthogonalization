\documentclass{article}
\usepackage[utf8]{inputenc}
\usepackage{amsmath,amsfonts}

% math definitions
\newcommand{\inner}[2]{\langle{#1},{#2}\rangle}

% fixing latex page layout and typography
\setlength{\textwidth}{5.50in}
\setlength{\textheight}{9.20in}
\setlength{\oddsidemargin}{3.25in}
\addtolength{\oddsidemargin}{-0.5\textwidth}
\setlength{\topmargin}{-0.20in}
\renewcommand{\small}{\normalsize} % pure evil
\linespread{1.08}
\frenchspacing\raggedbottom\sloppy\sloppypar
\pagestyle{myheadings}
\markboth{}{\textsf{Hogg \& Villar / Orthogonalization in relativity}}
\newcommand{\documentname}{\textsl{Note}}

\title{\bfseries Orthogonalization and vector operations in special and general relativity}
\author{\textbf{David W. Hogg}\\
        \textsl{Flatiron Institute}\\
        \and
        \textbf{Soledad Villar}\\
        \textsl{Johns Hopkins University}}
\date{June 2021}

\begin{document}\thispagestyle{plain}
\maketitle

\begin{abstract}\noindent
    Special and general relativity in $d+1$ dimensions (our macroscopic Universe is $3+1$) can be thought of as metric theories with a metric that isn't positive definite, such that timelike, spacelike, and lightlike vectors have positive, negative, and zero magnitudes.
    These spaces violate of a lot of the intuitions we have about subspaces, inner products, and orthogonality.
    The concept of orthogonalization (Gram--Schmidt, for example) carries over to Minkowski/Lorentz space, but there are pathologies in which it is possible for the procedure to produce a lightlike vector and thus bork.
    We describe how to avoid this and successfully orthogonalize any $d+1$ linearly independent input vectors; every successful such orthogonalization produces one time-like and $d$ space-like unit vectors, all orthogonal (in the Minkowski sense).
    We use these orthogonalizations to construct subspace projection operators.
    We use these, in turn, to construct Lorentz transformations that fix or vary particular vectors or subspaces.
    Like in many areas of physics, important aspects of relativity get simpler when we think of them in terms of unit vectors and inner products.
\end{abstract}

\section{Introduction}

...What are the metrics of sr and gr?

...Introduce concepts of O($d$) and O(1,$d$)...

...Remind what the LT is.

\section{Orthogonalization, projection, and rotation in O($d$)}

Before we consider special and general relativity, it is worth reviewing how orthogonalizations are performed and projection and rotation operators are constructed in ordinary space with an ordinary metric.
In standard $d$-dimensional space, with the standard Euclidean metric (the identity), containing vectors governed by the standard orthogonal group O($d$), inner products (scalar products) of vectors are defined as follows:
Given two column vectors $u,v\in\mathbb{R}^d$ (or, more specifically, $\mathbb{R}^{d\times1}$), the inner product $\inner{u}{v}\in\mathbb{R}$ is defined as
\begin{align}
    \inner{u}{v} &= u^\top v = v^\top u ~.
\end{align}
Two vectors $u,v$ are considered orthogonal if their inner product vanishes, or $\inner{u}{v}=0$.

....orthogonalization... Given a collection of $n\leq d$ linearly independent vectors $v_1,v_2,\ldots,v_n$, construct orthogonal vectors $u_1,u_2,\ldots,u_n$ and orthogonal unit vectors $\hat{u}_1,\hat{u}_2,\ldots,\hat{u}_n$ as follows:
\begin{align}
    u_1 &\leftarrow v_1
    \\
    \mbox{then for each $j$ ($2\leq j\leq n$) in order:} ~~ u_j &\leftarrow v_j - \sum_{k=1}^{j-1} \frac{\inner{v_j}{u_k}}{\inner{u_k}{u_k}}\,u_k
    \\
    \mbox{then for each $j$ ($1\leq j\leq n$):} ~~ \hat{u}_j &\leftarrow \frac{1}{\sqrt{\inner{u_j}{u_j}}}\,u_j
\end{align}

...projection matrices....

...rotation matrices...

\section{Notation}\label{sec:notation}

There is a history of notation in special and general relativity.
Here we deliver a translation from traditional Einstein summation notation, and traditional language about boost, to a more linear-algebra-oriented notation.
After this Section, we will be using exclusively the linear-algebra notation, which is simpler (for us).

In Lorentz or Minkowski space, we think of there being a metric $\Lambda$ (often called $g_{\mu\nu}$), which is a $(d+1)\times(d+1)$ matrix that is \emph{not} positive definite.
The metric $\Lambda$ is diagonal with a $+1$ in the first position and $-1$ repeated on all the remaining $d$ diagonal elements.
In $3+1$ this is
\begin{align}
    \Lambda &= \begin{bmatrix}1 & 0 & 0 & 0\\
                              0 & -1 & 0 & 0\\
                              0 & 0 & -1 & 0\\
                              0 & 0 & 0 & -1\end{bmatrix} ~.
\end{align}
Given two vectors $u$, $v$ in $d+1$, old-school relativists tend to write the relativistic inner product $\inner{u}{v}\in\mathbb{R}$ (the scalar product or Minkowski inner product) as
\begin{align}
    \inner{u}{v} &= u^\mu\,v_\mu = v^\mu\,u_\mu ~,
\end{align}
where $\mu$ is a component index\footnote{In this \documentname{}, greek indexes like $\mu$, $\nu$ will be indexes over vector components (going from 1 to 4 in $3+1$, for example), and roman indexes like $i$, $j$ will be indexes over vectors or other things.} going from 1 to $d+1$, and the repeated index is (implicitly) summed.
The $u^\mu$ is a contravariant vector component and the $v_\mu$ is a covariant vector component.
Contravariant and covariant components are related as follows:
\begin{align}
    u^\mu &= \Lambda^{\mu\nu}\,u_\nu \equiv \sum_{\nu=1}^{d+1} \Lambda^{\mu\nu}\,u_\nu
    \\
    \inner{u}{v} &= u^\mu\,v_\mu = \Lambda^{\mu\nu}\,u_\mu\,v_\nu \equiv \sum_{\mu=1}^{d+1}\sum_{\nu=1}^{d+1} \Lambda^{\mu\nu}\,u_\mu\,v_\nu
\end{align}
where the $\Lambda^{\mu\nu}$ are the components of the metric $\Lambda$.
The implicit summations on the left of the ``$\equiv$'' sign is called Einstein summation notation: Indexes can appear exactly once or exactly twice (and no more) and when they appear twice, one must be up and one must be down, and they are summed from 1 to $d+1$.

In linear algebra notation, if we think of $u$ and $v$ as being column vectors in $\mathbb{R}^{d+1}$ (or, to be extremely specific, $\mathbb{R}^{(d+1)\times 1}$), then we can write this same inner product as
\begin{align}
    \inner{u}{v} &= u^\top\Lambda\,v ~.
\end{align}
We are going to use this notation going forward, not the Einstein summation notation.

In standard special-relativity lore, Lorentz transformations are taught as \emph{boost transformations} in which the assignment of the stationary observer is changed and the time and space axes change accordingly.
For our purposes, the Lorentz transformations also include spatial rotations, spatial reflections, and even time reflections (gasp!):
That is, for our purposes, an operator $Q$ is a valid Lorentz transformation if it is a member of the group O(1,$d$).

An operator $Q$ (which can be thought of as a $(d+1)\times(d+1)$ matrix) is a Lorentz transformation if it preserves all inner products (scalar products). That is,
\begin{equation}
    Q \in \mbox{O(1,$d$)} ~ \mbox{if and only if} ~ \inner{u}{v}=\inner{Q\,u}{Q\,v} ~ \mbox{for all $u,v$ in $\mathbb{R}^{d+1}$} ~ .
\end{equation}
This, in turn, will be true if and only if $Q$ leaves the metric unchanged:
\begin{equation}
    Q \in \mbox{O(1,$d$)} ~ \mbox{if and only if} ~ Q^\top\Lambda\,Q=\Lambda ~ .
\end{equation}
This is our (surprising, perhaps) definition of the Lorentz transformation $Q$.
In the simplest case of $1+1$ ($d=1$), Lorentz transformations form a one-dimensional family; they are all of the form
\begin{align}
    Q &= \begin{bmatrix}\gamma & \beta\,\gamma \\ \beta\,\gamma & \gamma\end{bmatrix}
    \\
    \gamma &\equiv \frac{1}{\sqrt{1 - \beta^2}}
    \\
    \beta &\in (-1,1) ~.
\end{align}
In this simple case, $\beta$ is the dimensionless velocity of the boost, and $\gamma$ is the Lorentz factor.
There is more that can be said here, but for now, this defines the inner products and Lorentz transformations that will appear below.

\section{Relativistic orthogonalization}

\section{Relativistic projection operators}

\section{Vector-guided Lorentz transformations}

\section{Discussion}

What about spaces like O(2,5)?

\end{document}
